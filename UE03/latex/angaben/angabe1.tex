\begin{highlightblock}[DFT vs. FFT (10 Punkte)] \label{aufg:1}
In diesem Problem werden Sie die Ausführungszeiten von DFT und FFT Implementierungen vergleichen. Dazu sollen Sie zunächst in Matlab / Octave Funktionen schreiben, welche die DFT sowie die inverse DFT (IDFT) implementieren, indem Sie die jeweiligen Definitionsgleichungen umsetzen.

Schreiben Sie dann eine Funktion \m{eval_dft_vs_fft(N)}, welche zunächst ein komplexwertiges Zufallssignal der Länge $N$ erstellt. Vergleichen Sie anschließend die Ausführungszeiten ihrer Implementierungen mit den Matlab built-in Funktionen (FFT + IFFT), indem Sie das komplexwertige Signal zunächst transformieren (DFT bzw. FFT) und dann wieder rücktransformieren (IDFT bzw. IFFT). Das heißt Ausgangssignal sollte gleich dem Eingangssignal sein.

Um die Laufzeiten zu vergleichen verwenden Sie die Matlab Funktionen \m{tic} und \m{toc}. Zusätzlich sollen Sie den mittleren quadratischen Fehler (RMS error) zwischen den jeweiligen Eingangs- und Ausgangssignalen messen und vergleichen.
\end{highlightblock}