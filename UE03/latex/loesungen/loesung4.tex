%%%%%%%%%%%%%%%%%%%%%%%%%%%%%%%%%%%%%%%%%%%%%%%%%%%%%%%%%%%%
%%                       Lösung 4                         %%
%%%%%%%%%%%%%%%%%%%%%%%%%%%%%%%%%%%%%%%%%%%%%%%%%%%%%%%%%%%%

\newpage
\subsection{Modifikation des Programms}\label{aufg:4a}

Nach der Berechnung des Sinustons wird dieser in der \m{adsr_profile} Funktion verarbeitet und an das Musikstück angehängt.

\begin{lstlisting}[language=matlab]
% ...
s = sin(Omega*n);             % tone
sAdsr = adsr_profile(s);
music = [music sAdsr];        % concatenated audio signal
% ...
\end{lstlisting}

%%%%%%%%%%%%%%%%%%%%%%%%%%%%%%%%%%%%%%%%%%%%%%%%%%%%%%%%%%%%

\subsection{Grafische Ausgabe}\label{aufg:4b}

Für die Grafische Ausgabe wurden, um das Signal besser ablesen zu können, die Segmente je nach Frequenz unterschiedlich eingefärbt. 

\begin{figure}[h]
    \centering
    \includegraphics[width=1\linewidth]{assets/adsr1.png}
    \caption{Hüllkurven bewertetes Musiksignal}
    \label{fig:adsr1}
\end{figure}

%%%%%%%%%%%%%%%%%%%%%%%%%%%%%%%%%%%%%%%%%%%%%%%%%%%%%%%%%%%%

\subsection{Sampling Parameter}\label{aufg:4c}

Die Abtastfrequenz und die Tondauer stehen bei der Verarbeitung des Signals im Verhältnis zur Anzahl der Samples für jeden Ton. Die Gesamtlänge des Signals wird damit länger, wenn

\begin{itemize}
    \item die Sample-Rate vermindert, oder
    \item die Tondauer verlängert wird.
\end{itemize}

Die Zeiten der ADSR-Hüllkurve sind relativ zur Gesamtlänge angegeben. Ein Veränderung dieser Werte hat zur folge, dass die Einhüllende von jedem Ton eine andere Form hat, wie in Abbildung \ref{fig:adsr} gezeigt. Dabei ist zu beachten, dass $t_A+t_D+t_S<1$. Die Release-Zeit ist dann die verbleibende Zeit nach dem der Sustain abgelaufen ist. 

%%%%%%%%%%%%%%%%%%%%%%%%%%%%%%%%%%%%%%%%%%%%%%%%%%%%%%%%%%%%

\subsection{Harmonische Anteile}\label{aufg:4d}

Als Ziel wurde hier gewählt, den Ton einem Sägezahn anzunähern. Das kann erreicht werden, in dem weitere Sinusschwingungen mit einer ganzzahligen vielfachen Frequenz der Grundschwingung aufaddiert werden. Der Dämpfungsfaktor $\sfrac{1}{i}$ sorgt dafür, dass höhere Harmonische das Signal nicht verwüsten.

\begin{lstlisting}[language=matlab]
nHarmonics = 12;

for k=1:length(pitch)           
    % ...
    s = 0;
    for i = 1:nHarmonics
        s = 1/i * sin(i*Omega*n) + s; 
    end
    % ...
end
\end{lstlisting}

An einem stark vergrößertem Bild des Signals kann man das Sägezahn erkennen.

\begin{figure}[h]
    \centering
    \includegraphics[width=0.5\linewidth]{assets/saw.png}
    \caption{Sägezahn Signal}
    \label{fig:saw}
\end{figure}

%%%%%%%%%%%%%%%%%%%%%%%%%%%%%%%%%%%%%%%%%%%%%%%%%%%%%%%%%%%%

\stepcounter{subsection}
\subsection{Short-Time-Fourier-Transform}\label{aufg:4f}

\begin{figure}[h]
    \centering
    \includegraphics[width=0.8\linewidth]{assets/stft3d.png}
    \caption{3D STFT}
    \label{fig:stft3d}
\end{figure}

\newpage
\stepcounter{subsubsection}
\subsubsection{Vergleich Notenblatt}\label{aufg:4f2}

Speziell im Frequenzbereich der Grundschwingungen kann die Notenfolge gut erkannt werden.

\begin{figure}[h]
    \centering
    \includegraphics[width=0.9\linewidth]{assets/stft1.png}
    \caption{STFT von Unten}
    \label{fig:stft1}
\end{figure}
\begin{figure}[h]
    \centering
    \renewcommand{\it}{}
\begin{music}
\generalsignature1 % g maj / e min (sharp on f - see circle of fifths)
\smallmusicsize
\instrumentnumber{1}
\setstaffs1{1}
\generalmeter{\meterfrac44}
\sepbarrules
\startextract
% https://ctan.joethei.xyz/macros/generic/musixtex/doc/musixdoc.pdf
% high notes: abcdefg hijklmn (starting from h is an octave higher)
% low notes:  ABCDEFG HIJKLMN 
\notes \qu d \en
\bar
\Notes \qu g \en
\notes \Dqbu gh \en 
\Notes \ql{'b} \qu{'G} \en 
\bar
\NOtes \hl{'d} \qlp{b} \en
\Notes \cl{'b} \en
\bar
\notes \ql{'c} \Dqbl dc \Dqbl bc \ql d \en 
\bar
\notes \Dqbu hg \Dqbu hi \ql{'a} \en
\setdoubleBAR
\endextract
\end{music}
    \caption{Notenblatt zum Vergleich}
    \label{fig:notes}
\end{figure}

\subsubsection{FFT Länge reduzieren}\label{aufg:4f3}

Mit einer kleineren FFT-Länge wird es zunehmen schwieriger, die genaue Frequenz des Signals zu ermitteln - die Amplitude verschmiert. Im Gegenzug werden jedoch die zeitlichen Übergänge der Töne schärfer. 

\begin{figure}[h]
\begin{minipage}{0.5\textwidth}\vspace{0pt}
    \centering
    \includegraphics[width=1\linewidth]{assets/stft512.png}
    \caption{STFT 512}
    \label{fig:stft512}
\end{minipage}
\begin{minipage}{0.5\textwidth}\vspace{0pt}
    \centering
    \includegraphics[width=1\linewidth]{assets/stft256.png}
    \caption{STFT 256}
    \label{fig:stft256}
\end{minipage}
\end{figure}

Dieser Zusammenhang ist durch das Zeit-Bandbreite-Produkt gegeben \cite[p.~41]{GrünigenDanielCh.von2008DS:m}:

$$
\Delta t\cdot\Delta f =\text{const}, \quad \text{const} \geq \frac{1}{4\pi}
$$
