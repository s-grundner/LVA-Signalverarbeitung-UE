%%%%%%%%%%%%%%%%%%%%%%%%%%%%%%%%%%%%%%%%%%%%%%%%%%%%%%%%%%%%
%%                     Lösung Aufgabe 2                   %%
%%%%%%%%%%%%%%%%%%%%%%%%%%%%%%%%%%%%%%%%%%%%%%%%%%%%%%%%%%%%

\subsection{Impulsantwort}\label{aufg:2a}

Zum ermitteln der Impulsantwort $h[n]$ wird am Eingang des Systems ein Impuls angelegt.

$$ h[n] = -2\delta[n-1]+2\delta[n-2]-2\delta[n-3] $$

\begin{minipage}[t]{.5\textwidth}\vspace{0.8cm}
\centering
\plotsequence[1cm]{{1/0}}{$x[n]=\delta[n]$}
\end{minipage}%
\begin{minipage}[t]{.5\textwidth}\vspace{0pt}
\centering
\plotsequence[1cm]{{0/0, -2/1, 2/2, -2/3, 0/4}}{$y[n] = h[n]$}
\end{minipage}

Das System ist ein FIR-System, da es in einem endlichen Intervall klar bestimmt werden kann. Die Impulsantwort kann auch in der Form 

\begin{equation}
{\color{blue}h[n] = 2(-1)^n} \text{ für } n \in [1,3]
\end{equation}
\label{eq:a2a-h}

geschrieben werden.

\newpage

%%%%%%%%%%%%%%%%%%%%%%%%%%%%%%%%%%%%%%%%%%%%%%%%%%%%%%%%%%%%

\subsection{Frequenzantwort 1}\label{aufg:2b}

$$ H\left(e^{j\Omega}\right) := \sum\limits_{n=-\infty}^{\infty} h[n] e^{-j\Omega n} $$

Die Definitionsgrenzen für $h[n]$ können in die Summe eingesetzt werden.


$$
H\left(e^{j\Omega}\right) = \sum\limits_{n=1}^{3} 2(-1)^n e^{-j\Omega} = -2e^{-j\Omega} +2e^{-j2\Omega} -2e^{-j3\Omega}
$$

Aus der Eulerschen Formel (\ref{eq:euler}) und den Symmetrieeigenschaften von Sinus und Cosinus folgt:

\begin{align*}
&= &&-2 (\cos(-\Omega) +j\sin(-\Omega)) &&+2 (\cos(-2\Omega) +j\sin(-2\Omega)) && -2 (\cos(-3\Omega) +j\sin(-3\Omega)) \\
&= &&-2 (\cos(\Omega) -j\sin(\Omega)) &&+2 (\cos(2\Omega) -j\sin(2\Omega)) && -2 (\cos(3\Omega)  -j\sin(3\Omega))
\end{align*}

$$
\color{blue} H\left(e^{j\Omega}\right) = 2( \cos(2\Omega) - \cos(\Omega) - \cos(3\Omega) ) - j2( \sin(2\Omega) - \sin(\Omega) - \sin(3\Omega) )
$$

\begin{figure}[h]
\centering
\begin{minipage}[t]{.5\textwidth}\vspace{0pt}
    \includegraphics[width=\linewidth]{assets/A2b_AbsH.png}
    \caption{Betrag von $H\left(e^{j\Omega}\right)$}
    \label{fig:a2b-absH}
\end{minipage}%
\begin{minipage}[t]{.5\textwidth}\vspace{3pt}
    \includegraphics[width=1\linewidth]{assets/A2b_ArgH.png}
    \caption{Phase von $ H\left(e^{j\Omega}\right) $}
    \label{fig:a2b-argH}
\end{minipage}
\end{figure}

Wie man in Abbildung \ref{fig:a2b-argH} sehen kann, ist die Phase linear. Die Linearität der Phase ist eine Eigenschaft von FIR-Systemen mit spiegel- oder punktsymmetrischer Impulsantwort \cite[p.~217]{GrünigenDanielCh.von2008DS:m}. Dies ist zum Beispiel beim IIR-System (Abbildung \ref{fig:bode}) aus Aufgabe 2 nicht der Fall .

%%%%%%%%%%%%%%%%%%%%%%%%%%%%%%%%%%%%%%%%%%%%%%%%%%%%%%%%%%%%

\subsection{Frequenzantwort 2}\label{aufg:2c}

\begin{minipage}[t]{.5\textwidth}\vspace{0pt}

$$ H_1\left(e^{j\Omega}\right) = H\left(e^{j(\Omega+\pi)}\right) $$

Hier wird der Zusammenhang

$$
e^{j \theta_0 n} x[n] \quad \laplace \quad X\left(e^{j\left(\theta-\theta_0\right)}\right) 
$$

verwendet. Daher gilt für die Impulsantwort unter Anwendung der eulerschen Identität (\ref{eq:euler-ident})

$$ \color{blue} h_1[n] = e^{-j\pi n} h[n] = (-1)^n h[n] $$

\end{minipage}%
\begin{minipage}[t]{.5\textwidth}\vspace{3pt}
\centering
\plotsequence{{0/0, 2/1, 2/2, 2/3, 0/4}}{$h_1[n]$}
\end{minipage}

Es handelt sich nun um eine Rechteckfunktion.

%%%%%%%%%%%%%%%%%%%%%%%%%%%%%%%%%%%%%%%%%%%%%%%%%%%%%%%%%%%%

\subsection{Phase des Systems}\label{aufg:2d}

Die Frequenzantwort des linearen Systems ist lediglich eine Frequenzverschiebung eines anderen linearen Systems, weswegen die Phase ebenfalls linear ist.

\begin{figure}[h]
\centering
\begin{minipage}[t]{.5\textwidth}\vspace{0pt}
    \includegraphics[width=\linewidth]{assets/A2d_AbsH.png}
    \caption{Betrag von $H_1\left(e^{j\Omega}\right)$}
    \label{fig:a2d-absH}
\end{minipage}%
\begin{minipage}[t]{.5\textwidth}\vspace{3pt}
    \includegraphics[width=1\linewidth]{assets/A2d_ArgH.png}
    \caption{Phase von $ H_1\left(e^{j\Omega}\right) $}
    \label{fig:a2d-argH}
\end{minipage}
\end{figure}