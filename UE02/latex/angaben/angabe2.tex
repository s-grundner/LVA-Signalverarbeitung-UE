\begin{highlightblock}[LTI Systeme und DTFT (10 Punkte)]
Die Beziehung zwischen Eingangssignal $x[n]$ und Ausgangssignal $y[n]$ eines zeitdiskreten Systems ist durch
$$y[n] = -2x[n - 1] + 2x[n - 2] - 2x[n - 3]$$
gegeben.

\begin{enumerate}
\item[\ref{aufg:2a}] Bestimmen und skizzieren Sie die Impulsantwort des Systems. Ist das System ein FIR oder ein IIR System?
\item[\ref{aufg:2b}] Berechnen Sie die Frequenzantwort $H(e^{j\Omega})$ dieses Systems (d.h. die DTFT der Impulsantwort $h[n]$). Berechnen und skizzieren Sie qualitativ Betrag und Phase von $H(e^{j\Omega})$.

\textbf{Hinweis:} Verwenden Sie trigonometrische Zusammenhänge um einen einfachen Ausdruck für $H(e^{j\Omega})$ zu erhalten. Abgesehen von möglichen Phasensprüngen, verläuft die Phase linear? Falls ja, was bedeutet das für die Impulsantwort des Systems?
\item[\ref{aufg:2c}] Betrachten wir nun ein neues System mit der Frequenzantwort $H_1(e^{j\Omega}) = H(e^{j(\Omega + \pi)})$ Verwenden Sie eine Tabelle mit DTFT Eigenschaften um die die Impulsantwort $h_1[n]$ dieses Systems zu bestimmen und zu skizzieren. Welche Eigenschaft haben Sie verwendet?
\item[\ref{aufg:2d}] Handelt es sich noch immer um ein System mit linearer Phase? Ist das neue System mit der Impulsantwort $h_1[n]$ noch immer linear? (Sie müssen hier keinen Beweis liefern, geben Sie lediglich eine Erklärung dazu ab).
\end{enumerate}
\end{highlightblock}