Das zeitdiskrete Signal $x[n]$ ist periodisch mit der ganzzahligen Periode $N$, wenn gilt:

$$x[n] = x[n+N]$$

Das kleinste $N$, das die Gleichung erfüllt, ist die gesuchte Grundperiode. Für trigonometrische Funktionen muss ebenfalls die Periodizität gelten.

$$\sin{(\varphi(n))} = \sin{(\varphi(n)+2\pi k)}\quad\cos{(\varphi(n))} = \cos{(\varphi(n)+2\pi k)}, \quad k \in \mathbb{Z}$$

\subsection*{(a)}

\begin{align*}
\cos{\left(\frac{30}{105}\pi(n+N)\right)} &\overset{!}{=} \cos{\left(\frac{30}{105}\pi n + 2\pi k\right)} &&\Bigg|\arccos \\
\frac{30}{105}\pi(n+N) &\overset{!}{=} \frac{30}{105}\pi \left(n+ 2k\cdot\frac{105}{30}\right) &&\Bigg| \frac{105}{30\pi} \ \Bigg| -n \\
N&\overset{!}{=} \frac{105}{15}k = 7k
\end{align*}

Das kleinste ganzzahlige Ergebnis liefert $k=1\implies N=7$. In den folgenden Rechnungen wird die Forderung direkt an das Argument der Winkelfunktion gestellt.

\begin{minipage}[t]{0.49\textwidth}

\subsection*{(b)}

\begin{align*}
    0.02\pi (n+N) &\overset{!}{=} 0.02\pi n + 2\pi k\\
    \frac{1}{50}\pi(n+N) &\overset{!}{=} \frac{1}{50}\pi (n+100k) \\
    N &\overset{!}{=} 100k \implies N = 100
\end{align*}

\end{minipage}
\begin{minipage}[t]{0.49\textwidth}

\subsection*{(c)}

\begin{align*}
5(n+N) &\overset{!}{=} 5n + 2\pi k \\
5n+5N &\overset{!}{=} 5n + 2\pi k \\
5N &\overset{!}{=} 2\pi k
\end{align*}
\centering
$\pi$ ist irrational, daher ist \\
das Signal nicht periodisch.
\end{minipage}

\newpage

\begin{minipage}[t]{0.49\textwidth}
\subsection*{(d)}

\begin{align*}
5\pi (n+N) &\overset{!}{=} 5\pi n + 2\pi k \\
5\pi n+ 5\pi N  &\overset{!}{=} 5\pi n + 2\pi k \\
5\pi N &\overset{!}{=} 2k\pi \\
N &\overset{!}{=} \frac{2}{5}k \implies N = 2
\end{align*}

\end{minipage}
\begin{minipage}[t]{0.49\textwidth}
\subsection*{(e)}

\begin{align*}
\frac{62}{10}\pi (n+N) &\overset{!}{=} \frac{62}{10}\pi n + 2\pi k \\
\frac{62}{10}\pi n+\frac{62}{10}\pi N &\overset{!}{=} \frac{62}{10}\pi n + 2\pi k \\
N&\overset{!}{=} \frac{20}{62} k = \frac{10}{31}k \implies N = 10
\end{align*}

\end{minipage}

\vspace{10pt}

Veranschaulicht durch \m{stem} Plots:

\begin{figure}[ht]
    \centering
    \includegraphics[width=0.7\linewidth]{images/Signals.png}
    \caption{Signale}
    \label{fig:sig}
\end{figure}