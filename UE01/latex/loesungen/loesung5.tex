\subsection*{(a)}

$N_{ya} = N_x$: Ein- und Ausgang sind gleichgroß:

$$
\mathbf{y}_a = \mathbf{Ax}, \quad \mathbf{A} \in \mathbb{R}^{N_{ya}\times N_x}
$$

Die Matrix ist quadratisch. Einsetzen für $n$ in (\ref{eqn:1}) liefert:

\begin{align*}
y[0] &= a_0 x[0] \\
y[1] &= a_0 x[1] + a_1 x[0] \\
y[2] &= a_0 x[2] + a_1 x[1] + a_2 x[0]\\
& \ \ \vdots \\
y[N_{ya}-1] &= a_0 x[N_{ya}-1] + a_1 x[N_{ya}-2] + \dots + a_{N_{ya}-1} x[0]\\
\end{align*}

Eingesetzt in den Vektor $\mathbf{y}_a$:

$$
\begin{pmatrix}
    y[0] \\ y[1] \\ y[2] \\ \vdots \\ y[N_{ya}-1]
\end{pmatrix} = \left(\begin{array}{l}
    a_0 x[0] \\ a_1 x[0] + a_0 x[1] \\ a_2 x[0] + a_1 x[1] + a_0 x[2] \\ \qquad \vdots \\
    a_{N_{ya}-1} x[0] + \dots + a_1 x[N_{ya}-2] + a_0 x[N_{ya}-1]
\end{array}\right)
$$

\newpage

Durch erkennen der Linearkombination der Vektorelemente aus $\mathbf{x}$, lässt sich die Gleichung schreiben wie:

$$
\mathbf{y} = \underbrace{\begin{pmatrix}
    a_0 & 0 & 0 & \cdots & 0 \\
    a_1 & a_0 & 0 & \cdots & 0 \\
    a_2 & a_1 & a_0 & \cdots & 0 \\
    \vdots & \vdots & \vdots & \ddots & \vdots \\
    a_{N_{ya}-1} & a_{N_{ya}-2} & a_{N_{ya}-3} & \cdots & a_0 \\
\end{pmatrix}}_{\mathbf{A}} \mathbf{x}
$$

\subsection*{(b)}

Auch als Matlab-Skript im Anhang.

\begin{minted}{matlab}
function [ya,A] = mtxfilter(a,x)
    %calculate signal length and length of kernel
    sigLen = length(x);
    coeffLen = length(a);
    
    n = sigLen;
    m = sigLen+coeffLen-1;
    
    A = zeros(m,n);
    
    for i = 1:n
        A(i:i+length(a)-1,i) = a;
    end
    
    ya = A*x';
end
\end{minted}

\newpage

\subsection*{(c)}

\begin{figure}[h]
    \centering
    \includegraphics[width=1\linewidth]{images/ya.png}
    \caption{Eingangssignal $\mathbf{x}$, Impulsantwort von $\mathbf{A}$, Ausgangssignal $\mathbf{y}$ zu $\mathbf{x}$}
    \label{fig:ya}
\end{figure}

\newpage

\subsection*{(d)}

\begin{figure}[h]
    \centering
    \includegraphics[width=1\linewidth]{images/yb.png}
    \caption{Antwort auf $b[n]$}
    \label{fig:yb}
\end{figure}

\newpage

\subsection*{(e)}

Da die Eingänge lediglich addiert werden, lässt sich für die Matrix $\mathbf{C}$ einfach $\mathbf{A}$ und $\mathbf{B}$ addieren.

\begin{figure}[h]
    \centering
    \includegraphics[width=1\linewidth]{images/yc.png}
    \caption{Summe der Ausgänge, Filter mit $\mathbf{C}$, Absoluter Fehler}
    \label{fig:yc}
\end{figure}
