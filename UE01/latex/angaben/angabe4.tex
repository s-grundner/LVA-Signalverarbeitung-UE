\begin{highlightblock}[Impuls- und Sprungantwort einfacher LTI Systeme (10 Punkte)]
In diesem Besipiel betrachten wir 4 einfache diskrete Systeme.
\begin{enumerate}
\item Zeichnen Sie zunächst die Blockdiagramme der Systeme, welche über die folgenden Differenzengleichungen definiert sind:
\begin{enumerate}
\item Kammfilter: $y_K[n] = x[n]-x[n-4]$
\item Integrator: $y_I[n]=x[n]+y_I[n-1]$
\item Leaky Integrator: $y_{LI}[n] = Ax[n] + (1-A)y_{LI}[n-1], \quad A\in [0,1]$
\item Differenzierer: $y_D[n] = 0.5x[n] - 0.5x[n-2]$
\end{enumerate}
\item Skizzieren Sie nun die Impulsantworten $h[n]$ der 4 (kausalen) Systeme ($A=0.5$)
\item Speziell in der Regelungstechnik ist des öfteren auch die Sprungantwort $g[n]$ von Interesse. Diese kann über 2 äquivalente Definitionen beschrieben werden: Erstens, $g[n]$ ist die Antwort des Systems auf den Einheitssprung $u[n]$ (alle Eingangswerte sind $1$ für $n\geq 0$, sonst $0$. Eine zweite Definition lautet, dass $g[n]$ die kumulative Summe der Impulsantwort $h[n]$ ist. Mathematisch formuliert bedeutet das:

$$g[n]=\sum_{k = -\infty}^{n} h[k]$$

\end{enumerate}
Das bedeutet also, dass $g[n]$ zum Zeitindex $n$ gleich der Summe von allen vorigen Impulsantworten ist, bis inklusive der Impulsantwort $h[n]$. Skizzieren Sie, basierend auf diesem Wissen, die Sprungantworten der obigen $4$ (kausalen) Systeme ($A = 0.5$)
\end{highlightblock}