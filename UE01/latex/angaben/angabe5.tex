\begin{highlightblock}[Faltungsmatrix (12 Punkte)]
Die Faltung kann über die Differenzengleichung

\begin{equation}
y[n] = a_0x[n] + a_1x[n-1]+\dots a_{N_a-1}x[n-N_a+1]
\label{eqn:1}
\end{equation}

beschrieben werden. Der Eingang $x[n]$ und die konstante Koeffizienten-Sequenz $a[n]$ seien dabei $\mathbf{x}=[x[0],x[1],\dots, x[N_x-1]]^\mathrm{T}$ und $\mathbf{a}=[a[0],a[1],\dots,a[N_a-1]]^\mathrm{T}$. Die Ausgangssequenz sei definiert durch

$$\mathbf{y}_a = [y_a[0], y_a[1],\dots , y_a[N_{ya}-1]]^\mathrm{T}$$

wobei $N_{ya}=N_x$ gilt, d.h. (\ref{eqn:1}) wird bis zu einem maximalen $n=N_x$ ausgewertet. Es wird angenommen, dass $x[n] = 0$ wenn $n \notin [0,N_x-1]$, und $a[n]=0$ wenn $n\notin[0,N_a-1]$.

\begin{enumerate}
\item Gleichung (\ref{eqn:1}) kann auch in Matrix Form

\begin{equation}
\mathbf{y}_a= \mathbf{Ax}
\label{eqn:2}
\end{equation}

geschrieben werden, wobei $\mathbf{A}$ die sogenannte Faltungsmatrix ist. Diskutieren Sie die Struktur von $\mathbf{A}$. Was sind die resultierenden Dimensionen dieser Matrix? Geben Sie eine detaillierte Beschreibung für diese und die folgenden Fragen in Ihrem Protokoll.

\item Schreiben Sie eine \texttt{MATLAB} Funktion \m{[ya,A] = mtxfilter(a,x)} welche Gleichung (\ref{eqn:1}) implementiert und zusätzlich die Matrix $\mathbf{A}$ zurückgibt. Sie dürfen dabei die built-in Funktionen \m{convmtx} und \m{conv} \textbf{NICHT} verwenden! Die Funktion muss für Übergabeprameter \m{(a,x)} beliebiger Länge funktionieren.

\item Testen Sie Ihre Funktion mit dem Zeitsignal

$$x[n]=\sin{\left(2\pi\frac{8n}{N_x}\right)} + \sin{\left(2\pi \frac{16n}{N_x}\right)} + \sin{\left(2\pi\frac{80n}{N_x}\right)}, \quad n = 0,\dots, N_x-1$$

und den Filterkoeffizienten

$$
a[n] = \begin{cases}
(-1)^{n} \dfrac{\sin{(0.5\pi (n-D))}}{0.5\pi(n-D)} &\text{für } n=0,1,\dots, D-1, D+1,\dots, N_a-1\\
1 & \text{für } n = D 
\end{cases}
$$
Setzen Sie $N_x = 256$, $N_a=33$ und $D=16$ und führen Sie die folgenden Experimente aus:

Verwenden Sie Ihre Funktion \m{mtxfilter} um die Ausgangssequenz \m{ya[n]} zu berechnen. Visualisieren Sie ihre Resultate in einer Abbildung. Verwenden Sie dabei wiederum die Funktion \m{subplot}, um die Eingangs-Sequenz, die Impulsantwort und die Ausgangs-Sequenz in der gleichen Abbildung darzustellen. Verwenden Sie die Funktionen \m{xlabel} und \m{ylabel} um die Achsen korrekt zu beschriften.
\end{enumerate}
\end{highlightblock}

\begin{highlightblock}
\begin{enumerate}
\setcounter{enumi}{3}
\item Betrachten Sie nun eine weiter Filterfunktion gegeben als

$$b[n]=\delta[n-2]$$

mit $N_b=3$. Wiederholen Sie die Schritte aus der vorigen Aufgabe, indem Sie den Filter $b[n]$ verwenden und berechnen Sie $y_b[n]$

\item Summieren Sie die Signale $y_a[n]$ und $y_b[n]$ um das Ausgangssignal $y[n]$ zu erhalten. Berechnen Sie also

$$y[n]=y_a[n]+y_b[n]$$

und plotten Sie es in der Zeitdomäne. Wie können Sie die Matrizen $\mathbf{A}$ und $\mathbf{B}$ (welche von den Funktionen \m{[ya, A] = mtxfilter(a,x)} bzw. \m{[yb,B] = mtxfilter(b,x)} zurückgegeben werden) kombinieren, um eine neue Matrix $\mathbf{C}$ zu erhalten, sodass $\mathbf{y}=\mathbf{Cx}$? Berechnen und plotten Sie $\mathbf{y}'=\mathbf{Cx}$ zum Vergleich.

\end{enumerate}
\end{highlightblock}